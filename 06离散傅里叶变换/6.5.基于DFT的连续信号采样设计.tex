\section{基于DFT的连续信号采样设计}

本节简单介绍如何根据设计要求规划采样频率和采样量。

本节要点:
\begin{itemize}
    \item 掌握设计采样频率和采样量的方法。
\end{itemize}

~

对于连续信号的频率分析,如果使用计算机进行频域分析,首先得进行采样。
采样的两个关键性指标是采样频率$f_{sample}$和采样量$N$。

设计步骤:
\begin{enumerate}
    \item 确定一次完整信号的时间$t_0$。
    \item 确定信号中要考察的频段$f_L\text{~}f_H$,根据该频段决定采样频率,至少为上限的2倍,$f_{sample}=2f_H$。
    \item 确定采样量$N=f_{sample}\cdot t_0$。
    \item 据此可以得到DFT后的频域分辨率$\Delta f=f_{sample}/N=1/t_0$。
\end{enumerate}

注意:
\begin{itemize}
    \item 采样时间和频域分辨率是一对矛盾$\Delta f\cdot t_0=1$,采样时间越长,得到的DFT结果越细腻,如果系统设计需要考虑整体对分析结果的时间要求,则只能牺牲DFT细腻程度;
    \item 在采样时间一定的情况下,采样频率和采样量是一对矛盾$\frac{N}{f_{sample}}=t_0$,一般而言,硬件的ADC时间不能随意选择,而且如果用到芯片提供的FFT函数,采样量也无法选择,只能两者协调。
\end{itemize}

~

\begin{example}
假设一个信号时间为50ms,使用芯片开发包中的FFT库函数,库函数对一次FFT的数量要求为256、512或1024,分析DFT结果。
\end{example}

取512,有:
\begin{align*}
&\because t_0=0.05,N=512 \\
&\therefore \begin{cases}
	f_{sample}=\frac{N}{t_0}=10240\approx 10\mathrm{kHz}\\
	\Delta f=\frac{1}{t_0}=20\mathrm{Hz}\\
\end{cases}
\end{align*}
DFT的结果,频域范围为0~5kHz,单位频率为20Hz,同时需要芯片ADC的转化时间达到:
\[
t=\frac{t_0}{N}\approx 0.97\mathrm{\mu s}
\]




