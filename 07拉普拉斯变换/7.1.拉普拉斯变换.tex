\section{拉普拉斯变换}

本节介绍拉普拉斯变换。

本节要点:
\begin{itemize}
    \item 掌握拉普拉斯变换的定义;
    \item 了解拉普拉斯变换的性质。
\end{itemize}

%============================================================
\subsection{拉普拉斯变换的概念}

\begin{definition}[拉普拉斯变换]
若信号$x\left( t \right) $满足积分$\int_{-\infty}^{+\infty}{x\left( t \right) e^{-st}dt}$收敛,其中$s=\sigma +i\omega $为复数,则称该积分为{\bf $x\left( t \right) $的拉普拉斯变换}(Laplace Transform,LT),记为$X\left( s \right) $,即:
\[
X\left( s \right) =\int_{-\infty}^{+\infty}{x\left( t \right) e^{-st}dt}
\]
又称为{\bf 双边拉普拉斯变换}。
称满足积分收敛条件的$s$的定义域为{\bf 拉普拉斯变换的收敛域}。
同时将
\[
X\left( s \right) =\int_0^{+\infty}{x\left( t \right) e^{-st}dt}
\]
称为{\bf 单边拉普拉斯变换}。
若无特别指明,本笔记LT均指单边拉普拉斯变换。
相应地称
\[
x\left( t \right) =\frac{1}{2\pi i}\int_{c-i\infty}^{c+i\infty}{X\left( s \right) e^{st}ds}
\]
为{\bf 拉普拉斯逆变换},其中$c$为任意实数,且需满足$s=c+i\omega $始终在LT的收敛域内。
$x\left( t \right) $及其拉普拉斯变换形式通常记为:
\[
x\left( t \right) \overset{\mathscr{L}}{\leftrightarrow}X\left( s \right)
\]
\end{definition}

LT是一个自变量为复数的复函数。
LT相较于FT加入了一个收敛因子$e^{-\sigma}$,收敛域根据具体信号而定,不同的信号有不同的收敛域。
可以认为LT是广义的FT,只要是信号的LT收敛域包括$\sigma =0$,就有FT。
LT的优势在于对一个LTI系统,可以将其微分方程和初始条件一并变换成一个代数方程。

至此,我们对于信号或系统有了3个可供描述的空间:1)以微分方程和卷积为代表的时域;2)用FT得到的频域;3)用LT得到的s域。

~

\begin{example}
求信号$x\left( t \right) =e^{-bt}u\left( t \right) ,t\geqslant 0$的LT,其中$b\in \mathbb{R} $。
\end{example}

当$b+\sigma <0$时,有$\underset{t\rightarrow +\infty}\lim e^{-\left( b+s \right) t}=0$,则:
\[
X\left( s \right) =\frac{1}{b+s}
\]
收敛域是$\mathrm{Rs}\left[ s \right] >-b$。
特别地,当$b<0$时,信号有FT:
\[
X\left( \omega \right) =\frac{1}{b+i\omega}
\]

%============================================================
\subsection{拉普拉斯变换的性质}

{\bf 线性性}(由于积分的线性性,不证自明)
\[
ax\left( t \right) +by\left( t \right) \leftrightarrow aX\left( s \right) +bX\left( s \right)
\]

{\bf 时延性、频移性}(积分变量做一个变换即可证明)
\begin{align*}
&x\left( t-t_1 \right) u\left( t-t_1 \right) \leftrightarrow X\left( s \right) e^{-st_1} \qquad t_1>0 \\
&e^{s_1t}\cdot x\left( t \right) \leftrightarrow X\left( s-s_1 \right) \qquad s_1\in \mathbb{C}
\end{align*}

LT中的时移必须时延迟,即时间右移,左移是不存在的,因为LT的单边性使得左移后函数削掉了一部分。

{\bf 时展性}(积分变量做一个变换即可证明)
\[
x\left( at \right) \leftrightarrow \frac{1}{a}X\left( \frac{s}{a} \right) \qquad a>0
\]

LT中没有类似FT中的时间轴反转。

{\bf 三角律}(频移性结合欧拉公式即可证明)
\begin{align*}
&x\left( t \right) \cos \omega _1t\leftrightarrow \frac{1}{2}\left[ X\left( s+i\omega _1 \right) +X\left( s-i\omega _1 \right) \right] \\
&x\left( t \right) \sin \omega _1t\leftrightarrow \frac{i}{2}\left[ X\left( s+i\omega _1 \right) -X\left( s-i\omega _1 \right) \right]
\end{align*}

{\bf 时域的微分和积分}
\begin{align*}
&\frac{d^nx\left( t \right)}{dt^n}\leftrightarrow s^n\cdot X\left( s \right) -\sum_{i=1}^n{s^{n-i}\frac{d^{i-1}x\left( 0^- \right)}{dt^{i-1}}} \\
&\int_{-\infty}^t{x\left( \tau \right) d\tau}\leftrightarrow \frac{1}{s}\cdot X\left( s \right)
\end{align*}
特别地有:
\begin{align*}
&x'\left( t \right) \leftrightarrow s\cdot X\left( s \right) -x\left( 0^- \right) \\
&x''\left( t \right) \leftrightarrow s^2\cdot X\left( s \right) -s\cdot x\left( 0^- \right) -x'\left( 0^- \right)
\end{align*}

{\bf 频域的微分}
\[
t^n\cdot x\left( t \right) \leftrightarrow \left( -1 \right) ^n\frac{d^nX\left( s \right)}{ds^n}
\]

{\bf 卷积性}
\[
x\left( t \right) \ast y\left( t \right) \leftrightarrow X\left( s \right) Y\left( s \right)
\]

{\bf 初值定理}:已知信号$x\left( t \right) \leftrightarrow X\left( s \right) $,则其n阶导数的初值为:
\[
\left. \frac{d^nx\left( t \right)}{dt^n} \right|_{t=0^+}=\underset{s\rightarrow \infty}\lim \left[ s^{n+1}\cdot X\left( s \right) -\sum_{i=1}^n{s^{n+1-i}\frac{d^{i-1}x\left( 0^+ \right)}{dt^{i-1}}} \right]
\]
特别地有:
\begin{align*}
&x\left( 0 \right) =\underset{s\rightarrow \infty}\lim \left[ s\cdot X\left( s \right) \right] \\
&x'\left( 0 \right) =\underset{s\rightarrow \infty}\lim \left[ s^2\cdot X\left( s \right) -s\cdot x\left( 0 \right) \right]
\end{align*}

初值定理使得我们在已知$X\left( s \right) $而未知$x\left( t \right) $的情况,无需通过iLT直接求得$x\left( 0 \right) $。
其次,由于LT的单边性,我们得到是$x\left( 0 \right) $或$x\left( 0^+ \right) $,不是$x\left( 0^- \right) $!

{\bf 终值定理}:已知信号$x\left( t \right) $的LT为$X\left( s \right) $,如果$x\left( t \right) $存在,则有:
\[
\underset{t\rightarrow +\infty}\lim x\left( t \right) =\underset{s\rightarrow 0}\lim \left[ s\cdot X\left( s \right) \right]
\]

\begin{tcolorbox}
注意,LT没有FT中的“Duality”和“Parsevel定理”。
\end{tcolorbox}




