\section{LTI系统的差分方程模型}

本节讨论离散系统的差分方程模型。

本节要点:
\begin{itemize}
    \item 掌握差分方程的概念;
    \item 理解差分方程的物理意义;
    \item 掌握差分方程的求解思路;
    \item 编写Python函数求解n阶线性常系数差分方程。
\end{itemize}

%============================================================
\subsection{差分方程的概念}

如果离散LTI系统有限维度,可以用线性常系数差分方程描述如下:
\[
y\left[ n \right] +\sum_{i=1}^N{A_iy\left[ n-i \right]}=\sum_{i=0}^M{B_ix\left[ n-i \right]}
\]
即系统某刻的输出$y\left[ n \right] $,由此刻的输入$x\left[ n \right] $和之前的输入$x\left[ n-i \right] $和之前的输出$y\left[ n-i \right] $,一共$N+M+1$个量共同决定。
通常配合初始条件,通过递归计算的方法就可以解出$y\left[ n \right] $。
这样的系统称为{\bf 递归型离散系统}(recursive discrete-time system)或{\bf 递归型数字滤波器}(recursive digital filter)。

%============================================================
\subsection{差分方程的物理意义}

以一阶差分方程为例:
\[
y\left[ n \right] +Ay\left[ n-1 \right] =Bx\left[ n \right]
\]
表示系统当刻输出$y\left[ n \right] $是当刻输入$x\left[ n \right] $和前刻输出$y\left[ n-1 \right] $的线性叠加。

这里要注意:
\begin{itemize}
    \item $B$:对系统稳定性来讲没有任何关系,只是表明系统是放大器还是衰减器;
    \item $y\left[ n-1 \right] $:表示系统内部对能量的回响;
    \item $A$:表示这种回响以反馈的方式造成混响,决定了系统的稳定性。
\end{itemize}
还要特别注意,这种回响不是一次性的,它造成的叠加输出在后一刻继续以混响的方式反馈于系统。
如果系统是LTI,则这种回响是关于时间的一个等比数列,造成的混响就是这个等比数列的和。

%============================================================
\subsection{一阶差分方程的求解}

对于一阶差分方程$y\left[ n \right] +Ay\left[ n-1 \right] =Bx\left[ n \right] $,可以通过不断递归求解,这里直接给出解:
\[
y\left[ n \right] =\left( -A \right) ^ny\left[ 0 \right] +\sum_{i=1}^n{\left( -A \right) ^{n-1}Bx\left[ i \right]} \qquad n=1,2,\cdots
\]

%============================================================
\subsection{Python应用——n阶差分方程的实现}

这里,我们编写一个Python函数(my\_diff)用以求解n阶差分方程:
\[
y\left[ n \right] +\sum_{i=1}^N{A_iy\left[ n-i \right]}=B_0x\left[ n \right] +\sum_{i=1}^M{B_ix\left[ n-i \right]}
\]

已知(同时也是函数的参数):
\begin{itemize}
    \item 系数(按照差分方程的系数排序):
    \[
    A_1,A_2,\cdots ,A_N,B_0,B_1,B_2,\cdots ,B_M
    \]
    \item 初始条件(按时间从最远点到最近点排序):
    \[
    y\left[ -N \right] ,\cdots ,y\left[ -1 \right] ,x\left[ -M \right] ,\cdots ,x\left[ -1 \right]
    \]
    \item 输入信号序列:
    \[
    x\left[ 1 \right] ,x\left[ 2 \right] ,\cdots ,x\left[ n \right]
    \]
\end{itemize}
取$y\left[ 1 \right] ,y\left[ 2 \right] $观察:
\begin{align*}
y\left[ 1 \right] =&A_1y\left[ -N \right] +A_2y\left[ -N+1 \right] \cdots A_{N-1}y\left[ -2 \right] +A_Ny\left[ -1 \right] + \\
&B_1x\left[ -M \right] +B_2x\left[ -M+1 \right] \cdots B_{M-1}x\left[ -2 \right] +B_Mx\left[ -1 \right] + \\
&Bx\left[ 1 \right] \\
y\left[ 2 \right] =&A_1y\left[ -N+1 \right] +A_2y\left[ -N+2 \right] \cdots A_{N-1}y\left[ -1 \right] +A_Ny\left[ 1 \right] + \\
&B_1x\left[ -M+1 \right] +B_2x\left[ -M+2 \right] \cdots B_{M-1}x\left[ -1 \right] +B_Mx\left[ 1 \right] + \\
&Bx\left[ 2 \right]
\end{align*}

设计参数:
\begin{itemize}
    \item An:$N$维向量,表示$A_1,A_2,\cdots ,A_N$;
    \item Yn:$N$维向量,表示输出信号的初始值$y\left[ -1 \right] ,y\left[ -2 \right] ,\cdots ,y\left[ -N \right] $;
    \item Bm:$M$维向量,表示$B_1,B_2,\cdots ,B_M$;
    \item Xm:$M$维向量,表示输入信号的初始值$x\left[ -1 \right] ,x\left[ -2 \right] ,\cdots ,x\left[ -M \right] $;
    \item b:标量,表示$B_0$;
    \item X:表示输入信号$x\left[ 1 \right] ,x\left[ 2 \right] ,\cdots ,x\left[ n \right] $。
\end{itemize}

过程注解:
\begin{enumerate}
    \item 将Yn和Xm翻转,以匹配An和Bm;
    \item 构造输出Y,和X一样长度,填充0;
    \item 计算每个Y[i],期间更新Yn和Xm;
    \item 返回Y。
\end{enumerate}

注意:
\begin{itemize}
    \item An和Yn的长度必须一致;
    \item Bm和Xm的长度必须一致;
    \item b是标量;
    \item 最终输出信号Y的长度由输入信号X决定。
\end{itemize}

\begin{python}
def my_diff(
        An:np.ndarray, Yn:np.ndarray,
        Bm:np.ndarray, Xm:np.ndarray,
        b:float,
        X:np.ndarray
        ) -> np.ndarray:

    Yn = np.flipud(Yn)
    Xm = np.flipud(Xm)
    Y  = np.zeros_like(X)

    for i in range(X.shape[0]):
        Y[i] = b*X[i] + np.dot(Bm, Xm) - np.dot(An, Yn)
        Yn = np.append(Yn, Y[i])[1:]
        Xm = np.append(Xm, X[i])[1:]
        pass

    return Y
\end{python}




