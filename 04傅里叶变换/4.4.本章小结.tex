\section{本章小结}

傅里叶是频域分析的基础,必须深刻理解其含义,并熟练掌握计算方法。
时刻记住,数学本质上,傅里叶变换是函数的空间变换,将原本$x$空间的函数映射到三角函数(或指数函数)空间。
在信号与系统角度,具象化为时域到频域的变换,将原本对时间$t$的函数变换成对频率$\omega $的函数:
\[
\begin{array}{l}
	x=x\left( t \right)\\
	t\in \mathbb{R}\\
	x\in \mathbb{R}\\
\end{array} \quad \rightarrow \quad \begin{array}{l}
	X=X\left( \omega \right)\\
	\omega \in \mathbb{R}\\
	X\in \mathbb{C}\\
\end{array}
\]
一般来讲,$X\left( \omega \right) $的值域是复数,大小表示频率的振幅,角度表示频率的相位。

最后,需要特别注意$X\left( \omega \right) $的量纲$\mathrm{D}_x\cdot \mathrm{Hz}^{-1}$,及其物理意义“单位频率的信号,或者说信号的频率密度”。




