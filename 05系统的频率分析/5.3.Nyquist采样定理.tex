\section{Nyquist采样定理}

本节简单介绍Nyquist采样定理。

假设信号$x\left( t \right) $,以$T$周期采样,采样后信号记为$x_s\left( t \right) $,则有:
\[
x_s\left( t \right) =\sum_{n=-\infty}^{+\infty}{x\left( t \right) \delta \left( t-nT \right)}
\]
假设信号的傅里叶变换$X\left( \omega \right) $,则采样信号$x_s\left( t \right) $的傅里叶变换有:
\[
X\left( \omega \right) =\sum_{n=-\infty}^{+\infty}{\frac{1}{T}X\left( t-n\omega _s \right)}
\]
其中,$\omega _s=2\pi /T$称为{\bf 采样频率}。
若信号在频域的带宽$B$有限,且采样频率满足$\omega _s>2B$,则采样本身不会对信号有任何影响,而且采样后的信号可以通过低通进行完全恢复,这就是{\bf Nyquist采样定理}。

Nyquist采样定理的前提是信号是有限带宽信号。
但实际信号因为时域有限而导致频域无限,本身就不满足Nyquist定理的先决条件,所以无论采样频率多高,都会使信号的高频发生混叠。
通常的做法是先让信号通过一个低通去掉高频,再进行采样。




