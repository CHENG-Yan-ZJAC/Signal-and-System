\section{系统的阶跃响应和正弦响应}

本节讨论系统对两类特殊信号——阶跃信号和正弦信号——的响应。

本节要点:
\begin{itemize}
    \item 了解LTI系统的阶跃响应的组成部分;
    \item 了解LTI系统的正弦响应的组成部分。
\end{itemize}

%============================================================
\subsection{系统的阶跃响应}

设LTI系统的传递函数$H\left( s \right) =\frac{B\left( s \right)}{A\left( s \right)}$,对单位阶跃信号$u\left( t \right) $的响应:
\begin{align*}
&\because x\left( t \right) =u\left( t \right) \longleftrightarrow X\left( s \right) =\frac{1}{s} \\
&\therefore Y\left( s \right) =\frac{B\left( s \right)}{A\left( s \right)}\cdot \frac{1}{s}=\frac{E\left( s \right)}{A\left( s \right)}+\frac{H\left( 0 \right)}{s} \\
&\therefore y\left( t \right) =\mathscr{L} ^{-1}\left[ \frac{E\left( s \right)}{A\left( s \right)} \right] +H\left( 0 \right)
\end{align*}
可分为两部分:
\begin{itemize}
    \item {\bf 瞬态响应}(trancient)部分,记作$y_t\left( t \right) =\mathscr{L} ^{-1}\left[ \frac{E\left( s \right)}{A\left( s \right)} \right] $,根据特征根收敛或发散;
    \item {\bf 稳态响应}(stread-state)部分,记作$y_s\left( t \right) =H\left( 0 \right) $。
\end{itemize}
所以,对于任意LTI系统,可根据极点判断对阶跃信号的稳定性,若稳定,则稳定后系统输出为$H\left( 0 \right) $。

~

一阶LTI系统(即可用一阶微分方程描述的系统)的传递函数:
\[
H\left( s \right) =\frac{Q}{s+P}
\]
有且仅有一个实数极点,根据以上分析可得:
\begin{itemize}
    \item 一阶系统的阶跃响应不会有振荡形式;
    \item 如果极点$-P<0$,系统响应从0开始收敛增长到$H\left( 0 \right) =Q/P$;
    \item 如果极点$-P>0$,系统响应从0开始指数发散。
\end{itemize}
且$P$ 的大小决定了收敛或发散的速度。

二阶LTI系统(即可用二阶微分方程描述的系统)的传递函数:
\[
H\left( s \right) =\frac{Q}{s^2+Ps+R}
\]
\begin{itemize}
    \item 可以通过$s^2+Ps+R=0$判断系统的稳定性;
    \item 如果系统稳定,最终$H\left( 0 \right) =Q/R$。
\end{itemize}
如果系统稳定,则趋于稳定的方式有:
\begin{itemize}
    \item 实数极点,指数收敛(两个指数的叠加)到恒定值;
    \item 重复实数点,带 指数收敛到恒定值;
    \item 复数极点,必共轭,振荡收敛到恒定值,$P,R$决定了振幅和频率。
\end{itemize}

%============================================================
\subsection{系统的正弦响应}

假设LTI系统的传递函数为$H\left( s \right) =\frac{B\left( s \right)}{A\left( s \right)}$,当输入为正弦信号$x\left( t \right) =C\cos \left( \omega _0t \right) ,C\in \mathbb{R} ,\omega _0>0$时,系统响应:
\begin{align*}
&\because x\left( t \right) =C\cos \left( \omega _0t \right) \longleftrightarrow X\left( s \right) =\frac{Cs}{s^2+{\omega _0}^2} \\
&\therefore Y\left( s \right) =\frac{B\left( s \right)}{A\left( s \right)}\cdot \frac{Cs}{s^2+{\omega _0}^2}=\frac{E\left( s \right)}{A\left( s \right)}+\frac{\frac{C}{2}H\left( i\omega _0 \right)}{s-i\omega _0}+\frac{\frac{C}{2}\bar{H}\left( i\omega _0 \right)}{s+i\omega _0} \\
&\begin{aligned}
	\therefore y\left( t \right) &=\mathscr{L} ^{-1}\left[ \frac{E\left( s \right)}{A\left( s \right)} \right] +\frac{C}{2}H\left( i\omega _0 \right) e^{i\omega _0}+\frac{C}{2}\bar{H}\left( i\omega _0 \right) e^{-i\omega _0}\\
	&=\mathscr{L} ^{-1}\left[ \frac{E\left( s \right)}{A\left( s \right)} \right] +C\left| H\left( i\omega _0 \right) \right|\cos \left( \omega _0t+\angle H\left( i\omega _0 \right) \right)\\
\end{aligned}
\end{align*}
同样,响应可分为两部分,决定敛散性的{\bf 瞬态响应}(trancient)部分$y_t$和最终的{\bf 稳态响应}(stread-state)部分$y_s$:
\begin{align*}
&y_t\left( t \right) =\mathscr{L} ^{-1}\left[ \frac{E\left( s \right)}{A\left( s \right)} \right] \\
&y_s\left( t \right) =C\left| H\left( i\omega _0 \right) \right|\cos \left( \omega _0t+\angle H\left( i\omega _0 \right) \right)
\end{align*}
如果系统稳定,则输出为输入同频、变幅、延相正弦信号$y\left( t \right) =y_s\left( t \right) $。

%============================================================
\subsection{三类响应的对比}

我们将三类输入信号(冲激、阶跃、正弦)罗列在一起,如下表。

\begin{table}[h]
\centering
% \caption{表头}
\begin{tabular}{ccc}
    \toprule
    输入$x\left( t \right)$ & 输出$y\left( t \right)$\\
    \midrule
    $\delta \left( t \right) $ & $\mathscr{L} ^{-1}\left[ \frac{B\left( s \right)}{A\left( s \right)} \right] +0$\\
    $u\left( t \right) $ & $\mathscr{L} ^{-1}\left[ \frac{E\left( s \right)}{A\left( s \right)} \right] +H\left( 0 \right) $\\
    $C\cos \left( \omega _0t \right) $ & $\mathscr{L} ^{-1}\left[ \frac{E\left( s \right)}{A\left( s \right)} \right] +C\left| H\left( i\omega _0 \right) \right|\cos \left( \omega _0t+\angle H\left( i\omega _0 \right) \right) $\\
    \bottomrule
\end{tabular}
\end{table}

广义上讲,冲激响应也可以认为是瞬态响应和稳态响应的叠加,只不过稳态响应为0而已。
我们可以得出结论,对于因果LTI系统,如果系统稳定,则输出在经历一个过程后最终会和输入“一样”,只是会有增益和延时。




