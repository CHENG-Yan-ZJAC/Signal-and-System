\section{系统的冲激响应及稳定性}

本节讨论传递函数的用途之一——稳定性判断。

本节要点:
\begin{itemize}
    \item 充分理解系统响应函数和传递函数本质上的一致性;
    \item 理解稳定的定义;
    \item 理解稳定性判据;
    \item 掌握使用稳定性判据判断系统稳定性。
\end{itemize}

%============================================================
\subsection{系统的冲激响应及稳定性要求}

假设LTI系统的传递函数$H\left( s \right) $,对于单位阶跃信号$\delta \left( t \right) $的响应:
\begin{align*}
&\because \delta \left( t \right) \leftrightarrow 1 \\
&\therefore Y\left( s \right) =H\left( s \right) \cdot 1=H\left( s \right) \\
&\therefore y\left( t \right) =h\left( t \right)
\end{align*}
可见,LTI系统对于冲激信号的响应即为传递函数的iLT,两者是等价的。

\begin{definition}[系统的稳定性]
当一个LTI系统对于冲激信号,当时间足够长后,输出为0,我们称{\bf 系统稳定}(stable),如果输出不为0但有界,称系{\bf 统临界稳定}(marginally stable),如果输出发散,称{\bf 系统不稳定}(unstable)。
\[
\underset{t\rightarrow +\infty}\lim y\left( t \right) =\begin{cases}
	0 &\mathrm{stable}\\
	C &\mathrm{marginal} \,\, \mathrm{stable}\\
	\infty &\mathrm{unstable}\\
\end{cases}
\]
由于零状态LTI系统的冲激响应为$h\left( t \right) $,所以系统输出的稳定性也可以表示为冲激响应函数的敛散性:
\[
\underset{t\rightarrow +\infty}\lim h\left( t \right) =\begin{cases}
	0 &\mathrm{stable}\\
	C &\mathrm{marginal} \,\, \mathrm{stable}\\
	\infty &\mathrm{unstable}\\
\end{cases}
\]
\end{definition}

通俗来讲,就是当输入结束后,输出能不能降到0。

%============================================================
\subsection{系统的稳定性判据}

\begin{theorem}[LTI系统的稳定性判据]
假设零状态LTI系统有传递函数:
\[
H\left( s \right) =B\frac{\left( s-z_1 \right) \left( s-z_2 \right) \cdots \left( s-z_m \right)}{\left( s-p_1 \right) \left( s-p_2 \right) \cdots \left( s-p_n \right)} \qquad m<n
\]
则$H\left( s \right) $的极点反应了系统冲激响应的形状,具体来说有:
\begin{itemize}
    \item 对于实数单极点$p$,冲激响应包含增益项$ce^{pt}$;
    \item 对于复数共轭极点$p,\bar{p}$,冲激响应包含振荡项$2\left| c \right|e^{\sigma t}\cos \left( \omega t+\angle c \right) $;
    \item 对于重复极点,冲激响应包含$t$次方的多项式。
\end{itemize}
当$H\left( s \right) $所有极点满足$\mathrm{Re}\left[ p_i \right] <0$时,即均出现在零极图的左半边,对冲激信号的输出有$\underset{t\rightarrow \infty}\lim y\left( t \right) =0$,即系统为稳定系统。
当$H\left( s \right) $所有单极点满足$\mathrm{Re}\left[ p_i \right] \leqslant 0$,重复极点满足$\mathrm{Re}\left[ p_i \right] <0$时,称系统为临界稳定。
\end{theorem}

此判据证明过程略。

稳定系统对于冲激信号的输出一定会收敛到0。
临界稳定系统对于冲激信号的输出会稳定在一个非0值,或在一非0值振荡。
不稳定系统对于冲激信号的输出一定会发散到无穷大。

我们得到了传递函数的一个非常有用的用途——判断系统稳定性。
原本稳定性判断需要判断$\underset{t\rightarrow +\infty}\lim h\left( t \right) $的敛散性,这不但需要求解微分方程,而且还需要求极限。
但借助LT,我们可以通过考察传递函数的极点判断系统稳定性,方便了许多。

%============================================================
\subsection{劳恩——赫尔维茨判据}

上述稳定性判据需要求解系统传递函数的极点,即$A\left( s \right) =0$的根。
劳恩——赫尔维茨判据可以在不求解极点的情况下判断系统的稳定性。

判断步骤:
\begin{enumerate}
    \item 构造劳恩表;
    \item 第二列均大于0,表示系统绝对稳定;
    \item 第二列有0的项,表示系统临界稳定;
    \item 第二列有小于0的项,表示系统不稳定。
\end{enumerate}




