\section{系统的特性}

本节介绍系统的4大特性。

本节要点:
\begin{itemize}
    \item 掌握系统的4大特性:线性性,时不变,因果性,有限维度。
\end{itemize}

\begin{tcolorbox}
本书讨论的系统,都符合这4大特性。
\end{tcolorbox}

%============================================================
\subsection{线性和时不变}

若系统满足:
\begin{itemize}
    \item $x_1\left( t \right) +x_2\left( t \right) \rightarrow y_1\left( t \right) +y_2\left( t \right) $:称为{\bf 可加的}(additive),
    \item $ax\left( t \right) \rightarrow ay\left( t \right) $:称为{\bf 均匀的},或{\bf 齐次的}(homogeneous)。
\end{itemize}
若都满足,称为{\bf 线性的}(linear)。
严格来讲,讨论线性时,对系统的要求是无初始状态,即零状态。
若系统满足$x\left( t-t_0 \right) \rightarrow y\left( t-t_0 \right) $,称为{\bf 时不变}(time invariant)。
时不变表示系统的特性不随时间变化而变化,如RC电路,短时间看是一个时不变系统。
但长时间看,由于电容电解液的消耗导致电容值发生改变,是一个时变系统。

线性和时不变是系统最重要的两个特性。
如果系统满足线性和时不变,则称为{\bf LTI系统},这样的系统在计算上可以大大简化。
绝大多数自然界系统也都可以认为是LTI系统。
信号与系统这门学科着重研究的就是LTI系统的表述和分析,如何将一个系统建模成LTI,即便无法建模成LTI,那如何近似成一个LTI。

%============================================================
\subsection{记忆和因果}

若输出只与该时刻和之前的输入有关,称系统为{\bf 因果系统}(casual system)。
若系统的输出只取决于当刻的输入,则称为{\bf 无记忆系统}(memoryless system),如电阻,反之,称为{\bf 有记忆系统}(system with memory),如电容。
实际系统中,记忆是直接和能量相联系的,比如电容存储的电荷,车辆保有的动能。
无记忆系统都是因果系统。

能量在系统内部会有某种“惯性”,使得信号在系统内部造成“回响”效果,系统的输出就是之前各个时间点的回响的“混响”效应:
\[
y\left[ n \right] =\sum_{i=-\infty}^n{a_ix\left[ i \right]}=a_nx\left[ n \right] +\sum_{i=-\infty}^{n-1}{a_ix\left[ i \right]}=a_nx\left[ n \right] +y\left[ n-1 \right]
\]
这就是差分(同理也是微分)的由来。
时域模型就是研究系统的回响和混响。

%============================================================
\subsection{有限维度}

若系统的输入输出关系可以用微分方程
\[
y^{\left( n \right)}\left( t \right) = f\left[ y\left( t \right) ,y'\left( t \right) ,\cdots ,y^{\left( n-1 \right)}\left( t \right) , x\left( t \right) ,x'\left( t \right) ,\cdots ,x^{\left( m \right)}\left( t \right) ,t \right]
\]
描述,则称为{\bf 有限维度系统}(finite dimensional system),$n$为系统的{\bf 维度}(dimension)或{\bf 阶}(order)。
特别地,当有限维度系统可以描述为
\[
y^{\left( n \right)}\left( t \right) +\sum_{i=0}^{n-1}{a_i\left( t \right) y^{\left( i \right)}\left( t \right)}=\sum_{i=0}^m{b_i\left( t \right) x^{\left( i \right)}\left( t \right)}
\]
即为{\bf 线性系统}。
若更进一步,当输入输出的系数为常数,即
\[
y^{\left( n \right)}\left( t \right) +\sum_{i=0}^{n-1}{A_iy^{\left( i \right)}\left( t \right)}=\sum_{i=0}^m{B_ix^{\left( i \right)}\left( t \right)}
\]
时,为{\bf 线性时不变系统}。

对于离散系统,若输入输出满足差分方程
\[
y\left[ n \right] = f\left( y\left[ n-1 \right] ,\cdots ,y\left[ n-N \right] ,x\left[ n \right] ,x\left[ n-1 \right] ,\cdots ,x\left[ n-M \right] ,n \right)
\]
称为{\bf 有限维度系统}(finite dimensional system),同样地,$N$称为系统的{\bf 维度}(dimension)或{\bf 阶}(order)。

%============================================================
\subsection{其他特性}

系统的其他特性还包括{\bf 可逆性}。
如编码系统,必须是可逆的,必有且仅有一个与之对应的逆系统进行解码。
如果系统在一个有限的输入下,响应收敛,称为{\bf 稳定的}。
一般来讲稳定性是由于能量消耗的原因。




