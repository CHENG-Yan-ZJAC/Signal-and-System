\section{信号与系统概述}

本节介绍信号和系统的概念。

本节要点:
\begin{itemize}
    \item 掌握信号、系统及其相关概念。
\end{itemize}

%============================================================
\subsection{信号的概念}

\begin{definition}[信号]
在数学的角度,{\bf 信号}(signal)就是函数,自变量可以是时间、空间等,可以是一元函数,也可以是多元函数。
本书讨论的信号是以时间为自变量的一元函数,通常记为$x\left( t \right) $。
\end{definition}

物理意义上,信号指的是一个物体的状态。
若物体是一实物,则状态可以是物体的位移、速度。
若物体是一场,则状态可以是场强、电压、电流。
总之,信号描述的是一个物体的状态随时间的变化。

可以定义信号的{\bf 能量}和{\bf 功率}:
\begin{align*}
&E=\int_{t_1}^{t_2}{\left| x\left( t \right) \right|^2dt} \\
&P=\frac{1}{t_2-t_1}\int_{t_1}^{t_2}{\left| x\left( t \right) \right|^2dt}
\end{align*}
从能量和功率的角度,信号可以分有限能量信号$E_{\infty}<\infty $、有限功率信号$P_{\infty}<\infty $和能量功率均无限的信号。
而且易得,如果有限能量必然功率为0,如果有限功率必然能量无穷。

这个“功率”和“能量”是纯粹信号与系统上的概念。
虽然和具体物理意义无关,但还是有抽象层面的关系,如弹性势能$E=\frac{1}{2}kl^2$,动能$E=\frac{1}{2}mv^2$。

~

信号与系统中一个重要的概念就是信号的变换,对于以时间为自变量的信号指的就是时间轴的变换:
\begin{itemize}
    \item {\bf 时移}(time shift):$x\left( t-t_0 \right) $,表示信号延迟,或者波形右移。
    \item {\bf 尺变}(time scaling):$x\left( at \right) $,$a>1$表示信号变快,波形压缩,$a<1$表示信号变慢,波形伸长,$a=-1$表示信号{\bf 反转}(time reversal)。
    \item {\bf 周期}:若信号有$x\left( t+T \right) =x\left( t \right) $,则称为{\bf 周期信号}(periodic signal),最小正值$T$称为{\bf 基波周期}(fundamental period)。
    \item {\bf 奇偶性}:若信号有$x\left( t \right) =x\left( -t \right) $,称为{\bf 偶信号}(even),若$x\left( t \right) =-x\left( -t \right) $,称为{\bf 奇信号}(odd),易得,所有信号都可以分解为奇信号和偶信号的和:
    \begin{align*}
    &x_{even}\left( t \right) =\frac{1}{2}\left[ x\left( t \right) +x\left( -t \right) \right] \\
    &x_{odd}\left( t \right) =\frac{1}{2}\left[ x\left( t \right) -x\left( -t \right) \right]
    \end{align*}
\end{itemize}

\begin{tcolorbox}
注,$x\left( t \right) =C$是周期信号,但没有基波周期。
\end{tcolorbox}

%============================================================
\subsection{系统和响应的概念}

\begin{definition}[系统]
{\bf 系统}(system)是组件与端口的互连,通过这些端口可以输出输入信息。
\end{definition}

我们称一组描述输入输出之间相互关系的数学方程为{\bf 系统的数学模型}(mathematical model)。
数学模型是一种描述系统的数学工具,好的数学模型是需要在精确性和简单性之有良好的平衡,两大基本数学模型:
\begin{itemize}
    \item {\bf 输入输出模型}(input/output model):描述输入信号和输出信号之间关系,具体分:
    \begin{itemize}
        \item {\bf 微分方程}(differential equation)和{\bf 差分方程}(difference equation),
        \item {\bf 卷积模型}(convolution model),
        \item {\bf 傅里叶变换}(Fourier transform),
        \item {\bf 传递函数}(transfer function);
    \end{itemize}
    \item {\bf 状态模型}(state model):描述输入、状态、输出之间关系。
\end{itemize}
其中,傅里叶变换可视为传递函数的一个特例。

\begin{definition}[响应]
对于一个特定的输入或系统本身状态,系统会产生一个特定的输出,称为{\bf 响应}(response)。根据引起响应的不同原因,响应可以分为:
\begin{itemize}
    \item {\bf 零输入响应}(zero input):系统响应仅由其初始状态引起,输入始终为0;
    \item {\bf 零状态响应}(zero state):系统初始状态为0,响应仅由输入引发。
\end{itemize}
\end{definition}

\begin{tcolorbox}
为简化讨论,本笔记讨论只有一个输入和一个输出的系统,且系统默认都是零状态响应。
\end{tcolorbox}







